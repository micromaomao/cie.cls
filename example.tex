\documentclass{cie}
\usepackage{lipsum}
\usepackage{ccicons}
\begin{document}
  % \printanswer
  \copyright{Mao Wtm \ccbysa}
  \subject{SOCIOLOGY}
  \syllabus{9699}
  \component{13}
  \componentfull{Paper 1: Lorem Ipsum}
  \candidateinfoboxes{}
  \additionmaterials{
    Additional materials: CIE Answer Sheet
  }
  \answeron{}
  \writinginstructions{
    Write your \textbf{name and class} on \textbf{all} the work you hand in. \\
    Write in \textbf{pen or pencil}, but not lighter than HB. \\
    Although the use of staples, paper clips, glue or correction fluid won't cause issues, we advise against using them in exams. \\
    Leave a blank line between each questions. \\
  }
  \testinginstructions{
    Answer question \ref{qsa} and \textbf{either} question \ref{qsb1} or question \ref{qsb2}. \\
  }
  \coverpage

  \section{}{
    \ifmarkscheme Candidates should a\else A\fi nswer question \ref{qsa}
  }

  \begin{question}%
    \label{qsa}%
    \ifnotmarkscheme \lipsum[2-3] \par \fi%
    \subquestion Define the term ``Sed diam''. \point{2}
    \markschemecontent{
      \award{1} for a partial definition of ``Sed diam''. \\
      \award{1} for a full definition of ``Sed diam''.

      An example on its own which do not aid the definition should not be awarded the mark.
    }

    \subquestion Describe two example of consequat lorem. \point{4}
    \markschemecontent{
      \award{1} for identifying one example, and \award{1} for the explaination. \\
      Another \award{1} for identifying another example, and \award{1} for the explaination.

      Development can be either by a description of the consequat lorem or the effects of
      consequat lorem.
    }

    \subquestion Explain why Vestibulum luctus nibh at lectus diam. \point{8}
    \markschemecontent{
      \award{0}-\award{4}: Answers at this level are likely to show only limited
      appreciation of the issues raised in the question. 

      Lower in the level \award{1}–\award{2}, a simple answer with no development. 

      Higher in the level \award{3}–\award{4}, a few limited observations, but with little
      depth in the explanations offered and the answer may rely on description rather
      than explanation. Answers which implicitly link to research or methods may
      reach the top of the level. 

      \award{5}-\award{8}: Answers at this level show some sociological knowledge and
      understanding of the question. At this level there is likely to be some
      accurate use of studies, concepts, contemporary examples, or some explicit
      discussion of the issue in question.

      Lower in the level \award{5}–\award{6}, a simplistic description is expected.

      Higher in the level \award{7}–\award{8}, a more detailed account of some
      different reasons for why Vestibulum luctus nibh at lectus diam.

      Place at the top of the level according to depth and/or range of examples
      explained and supported with theory, empirical data or concepts.

      A good list of undeveloped points may gain up to 6 marks. To go higher there
      needs to be some development of three or more points or detailed development of
      two or more points.

      This question asks candidates to ‘explain’ therefore there is no requirement
      for assessment.
    }

    \subquestion Assess the view `Class aptent taciti sociosqu ad litora torquent per conubia nostra,
      per inceptos hymenaeos.'. \point{11}

    \markschemecontent{
      \award{0}-\award{4}: Answers at this level are likely to show only limited
      appreciation of the issues raised in the question. 

      Lower in the level \award{1}–\award{2}, a simple answer with no development. 

      Higher in the level \award{3}–\award{4}, a few limited observations, but with little
      depth in the explanations offered and the answer may rely on description rather
      than explanation. Answers which implicitly link to research or methods may
      reach the top of the level. 

      \award{5}-\award{8}: Answers at this level show some sociological knowledge and
      understanding of the question. At this level there is likely to be some
      accurate use of studies, concepts, contemporary examples, or some explicit
      discussion of the issue in question.

      Lower in the level \award{5}–\award{6}, one or two simplistic descriptions
      (e.g. of consumption/subservience compared to either that of the functionalist
      or the feminist views). Answers are likely to be supported with brief
      references to studies. Higher in the level \award{7}–\award{8}, a more detailed
      account supported with more detailed references to studies (e.g. Lorem
      ipsum dolor sit amet, consectetuer adipiscing elit. Ut purus elit,
      vestibulum ut, placerat ac, adipiscing vitae, felis.)

      Place at the top of the level according to depth and/or range of examples
      explained and supported with theory, empirical data or concepts.

      Answers in this level should address both sides of the debate but a one-sided
      answer that is done very well, could also gain up to 8 marks.

      \award{9}-\award{11}: Answers at this level must achieve three things: 

      \begin{itemize}
        \item [\textbf{First},] there will be good sociological knowledge and understanding.  

        \item [\textbf{Second},] the material used will be interpreted accurately and applied
        effectively to answering the question. 

        \item [\textbf{Third},] there must also be some evidence of assessment.
      \end{itemize}

      Lower in the level \award{9}–\award{10}, the assessment may be based on a
      simple juxtaposition of different views of how Class aptent taciti sociosqu ad
      litora torquent per conubia nostra, per inceptos hymenaeos. Alternatively
      answers may be confined to just one or two explicitly evaluative points. At the
      top of the level \award{11}, the taciti sociosqu view that sociosqu ad litora
      torquent per conubia nostrawill be evaluated explicitly and in some depth
      and/or with wider range of explicitly evaluative points.  The taciti sociosqu
      view will be explored, probably through a discussion of key concepts. This
      should be evaluated directly, mostly likely through clear and explicit
      understanding of aspects of the taciti sociosqu view that sociosqu ad litora
      torquent per conubia nostrawill, but at this level it may be through comparison
      with other revalent theories. 
    }

    \printtotalmark
  \end{question}

  \section{}{
    \ifmarkscheme Candidates should a\else A\fi nswer either question \ref{qsb1} or \ref{qsb2}.
  }

  \begin{question}%
    \label{qsb1}%
    Explain and assess the view ``there is no one who loves pain itself, who seeks after it and wants to
    have it, simply because it is pain.'' \pointnoadd{25}
  \end{question}

  \begin{question}%
    \label{qsb2}%
    Explain and assess the view ``Neque porro quisquam est qui dolorem ipsum quia dolor sit amet,
    consectetur, adipisci velit.'' \pointnoadd{25}
  \end{question}

  \addtopapermark{25}
  \finalstuff
\end{document}
