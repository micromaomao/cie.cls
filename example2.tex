\documentclass{cie}
\usepackage{lipsum}
\usepackage{amsmath}
\usepackage{siunitx}
\usepackage{ccicons}
\usepackage{commath}
\begin{document}
  % \printanswer
  \copyright{Mao Wtm \ccbysa}
  \subject{MATHEMATICS}
  \syllabus{9709}
  \component{43}
  \componentfull{Paper 4: Also Pure Mathematics}
  \season{October/November 2017}
  \coverpage

  \begin{question}%
    It is given that $\frac{1}{\infty} = 0$.

    \subquestion Prove $-8 = 10$. \point{2}
    \writinglinesfor{5cm}
    \markschemecontent{
      \award{1} for attempts at rotation, e.g: $-18 = 0$ \\
      \award{1} for complete and correct proof.

      Rotate both side \SI{90}{\degree} counter-clockwise: $-18 = 0$ \\
      Divide both side by 1: $-8 = \frac{0}{1}$ \\
      In right hand side, dividing by 1 is the same as multiply by 1: $-8 = 10$
    }

    \subquestion Hence prove $\frac{1}{0} = \infty$. \point{2}
    \writinglinestillvfill
    \markschemecontent{
      \award{1} for attempts at multiplying both side by $-1$: $8 = -10$ \\
      \award{1} for complete and correct proof: Rotate both side \SI{90}{\degree} clock-wise: $\infty = \frac{1}{0}$
    }

    \printtotalmark
    \ifnotmarkscheme \newpage \fi
  \end{question}

  \begin{question}%
    \subquestion Prove the identity ``$\text{opinion} - \pi = \text{onion}$''. \point{1} 
    \writinglinesfor{3cm}
    \markschemecontent{
      Rewrite $\pi$ as $\text{pi}$, and removing $\text{pi}$ from $\text{o\textbf{pi}nion}$ gives $\text{onion}$.
    }

    \subquestion Prove the identity $\frac{7}{s} = 0 \mod 2$. \point{3}
    \writinglinesfor{6cm}
    \markschemecontent{
      \award{1} for rewriting $7$ as $seven$. Do \textbf{not} award the mark if the candidate wrote seven in a sans-serif font. \\
      \award{1} for dividing $s$ from $seven$, giving $even$. \\
      \award{1} stating that $even$ implies $0 \mod 2$.
    }

    \subquestion Explain why obtuse triangles are always sad. \point{1}
    \writinglinestillvfill
    \markschemecontent{
      Because it is never right.
    }

    \printtotalmark
    \ifnotmarkscheme \newpage \fi
  \end{question}

  \begin{question}%
    \subquestion Prove the following identity: \point{3}

    $$\frac{1}{8} \left (4 \cos \left (4 \theta + \sqrt{ \sin \left (\frac{\theta}{2} \right ) } \right ) + \cos \left (8 \theta + 2 \sqrt{ \sin \left (\frac{\theta}{2} \right ) } \right ) + 3 \right ) =
      \cos^4 \left (2 \theta + \frac{1}{2} \sqrt{ \sin \left (\frac{\theta}{2} \right ) } \right )$$

    \writinglinestillvfill
    \markschemecontent{
      \award{3} for correct proof.
    }
    \ifnotmarkscheme \newpage \fi

    \subquestion Hence solve $\frac{1}{8} \left (4 \cos \left (4 \theta + \sqrt{ \sin \left (\frac{\theta}{2} \right ) } \right ) + \cos \left (8 \theta + 2 \sqrt{ \sin \left (\frac{\theta}{2} \right ) } \right ) + 3 \right )
      = \tan \theta$
      for $0 < x < 2\pi$. \point{3}
    \writinglinesfor{8cm}
    \markschemecontent{
      Solutions are $\theta \approx 0.2807$ and $\theta \approx 3.3216$.

      \award{1} for attempting to use change-of-sign with $\cos^4 \left (2 \theta + \frac{1}{2} \sqrt{ \sin \left (\frac{\theta}{2} \right ) } \right ) - \tan \theta = 0$ \\
      \award{1} for one correct solution. \\
      \award{1} for another correct solution.
    }

    \printtotalmark
  \end{question}

  \begin{question}%
    It is given that John has 3 apples, and he then lost 1 apple.

    \subquestion Show that John has 2 apple left. \point{1}
    \writinglinesfor{2cm}
    \markschemecontent{
      \award{1} for correct application of subtraction.
    }

    \subquestion Hence calculate the mass of the Sun. \point{3}
    \writinglinestillvfill
    \ifnotmarkscheme \newpage \fi
    \markschemecontent{
      \award{1} for attempting to calculate mass from volume (\SI{1.412e27}{m^3}) and average density (\SI{1408}{kg/m^3}).
      \award{1} for correct answer and \award{1} for correct unit: \SI{1.988e30}{kg}
    }

    It is given that $\SI{1000}{gram} = \SI{1}{kilogram}$.

    \subquestion Write 1 instagram in \SI{}{gram}. \point{1}
    \writinglinesfor{3cm}
    \markschemecontent{
      $$\int \SI{1}{instagram} \dif t$$

      Comment: Since instagrams are a snapshot of grams at a specific point in time, to convert grams to instagrams you simply need to take the derivative.
      To convert instagrams back into grams you use instagration. See \url{https://www.reddit.com/r/shittyaskscience/comments/1dicsm/whats_the_conversion_formula_for_grams_to/}
    }

    \printtotalmark
  \end{question}

  \finalstuff
\end{document}
