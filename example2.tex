\documentclass{cie}
\usepackage{lipsum}
\usepackage{amsmath}
\usepackage{siunitx}
\usepackage{ccicons}
\begin{document}
  \copyright{Mao Wtm \ccbysa}
  \subject{MATHEMATICS (FUNNIER)}
  \syllabus{2333}
  \component{13}
  \componentfull{Paper 1: Pure Mathematics}
  \coverpage

  \begin{question}
    It is given that $\frac{1}{\infty} = 0$.

    \subquestion Prove $-8 = 10$. \point{2}
    \writinglinesfor{5cm}

    \subquestion Hence prove $\frac{1}{0} = \infty$. \point{2}
    \writinglinestillvfill
    \printtotalmark
    \newpage
  \end{question}

  \begin{question}
    \subquestion Prove the identity ``$\text{opinion} - \pi = \text{onion}$''. \point{1} 
    \writinglinesfor{3cm}

    \subquestion Prove the identity $\frac{7}{s} = 0 \mod 2$. \point{2}
    \writinglinesfor{6cm}

    \subquestion Explain why obtuse triangles are always sad. \point{2}
    \writinglinestillvfill
    \printtotalmark
    \newpage
  \end{question}

  \begin{question}
    \subquestion Prove the following identity: \point{3}

    $$\frac{1}{8} \left (4 \cos \left (4 \theta + \sqrt{ \sin \left (\frac{\theta}{2} \right ) } \right ) + \cos \left (8 \theta + 2 \sqrt{ \sin \left (\frac{\theta}{2} \right ) } \right ) + 3 \right ) =
      \cos^4 \left (2 \theta + \frac{1}{2} \sqrt{ \sin \left (\frac{\theta}{2} \right ) } \right )$$

    \writinglinestillvfill
    \newpage

    \subquestion Hence solve $\frac{1}{8} \left (4 \cos \left (4 \theta + \sqrt{ \sin \left (\frac{\theta}{2} \right ) } \right ) + \cos \left (8 \theta + 2 \sqrt{ \sin \left (\frac{\theta}{2} \right ) } \right ) + 3 \right )
      = \tan \theta$
      for $0 < x < 2\pi$. \point{3}
    \writinglinesfor{8cm}

    \printtotalmark
  \end{question}

  \begin{question}
    It is given that John has 3 apples, and he then lost 1 apple.

    \subquestion Show that John has 2 apple left. \point{1}
    \writinglinesfor{2cm}

    \subquestion Hence calculation the mass of the Sun. \point{3}
    \writinglinestillvfill
    \newpage

    It is given that $\SI{1000}{gram} = \SI{1}{kilogram}$.

    \subquestion Write 1 instagram in \SI{}{gram}. \point{1}
    \writinglinesfor{3cm}

    \printtotalmark
  \end{question}

  \finalstuff
\end{document}
